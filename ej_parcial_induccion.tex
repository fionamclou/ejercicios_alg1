\documentclass{article}
\usepackage{amsmath}
\usepackage{amssymb}  % Paquete para símbolos adicionales

\begin{document}

\section*{Ejercicio recuperatorio primer parcial 2°C 2024}

4. Demostrar que, para todo \( n \in \mathbb{N} \), \( n \geq 2 \),
\[
\binom{2n}{n} \leq 3 \cdot 2^{2n - 3}
\]

\noindent\rule{\textwidth}{0.4pt} % Línea separadora

Este ejercicio se ve mas díficil de lo que es. Solo tenes que recordar que la fórmula general del número combinatorio es:
\[
\binom{n}{k} = \frac{n!}{k!(n-k)!}
\]
Despues sale bastante directo.

\subsection*{Proposición:}
\[
p(n) : \binom{2n}{n} \leq 3 \cdot 2^{2n - 3}
\]


\subsection*{Caso base:}
Probaremos para \( n = 2 \). En este caso:
\[
\binom{2 \cdot 2}{2} \leq 3 \cdot 2^{2 \cdot 2 - 3}
\]
\[
\frac{4!}{2! \cdot 2!} \leq 6 
\]
\[
\ 6 \leq 6 
\] \(\checkmark\) \ 
Se cumple el caso base.

\subsection*{Paso inductivo:}
Sea  \( k \in \mathbb{N}\) dado. Supongo p(k) es verdadero, quiero probar que p(k+1) es verdadero.


\subsection*{HI:}
Supongo p(k) verdadero.
\[
\binom{2k}{k} \leq 3 \cdot 2^{2k - 3}.
\]
Es decir
\[
\frac{(2k)!}{(2k-k)! \, k!} \leq 3 \cdot 2^{2k-3}.
\]
\[
\frac{(2k)!}{k! \, k!} \leq 3 \cdot 2^{2k-3}.
\]

\subsection*{QPQ:}
Quiero probar que p(k+1) es verdadero. 
\[
\binom{2(k+1)}{k+1} \leq 3 \cdot 2^{2(k+1) - 3}.
\]

\subsection*{Trabajo con lo que sé}
La idea es ir trabajando desde lo que quiero probar con la fórmula del número combinatorio, hasta llegar a la HI.

\[
\binom{2(k+1)}{k+1} \leq 3 \cdot 2^{2(k+1) - 3}.
\]  \[
\binom{2k+2}{k+1} \leq 3 \cdot 2^{2k-1}.
\] 
Aplico fórmula general del número combinatorio.
\[
\frac{(2k+2)!}{(2k+2-k-1)! \, (k+1)!} \leq 3 \cdot 2^{2k-1}.
\]
\[
\frac{(2k+2)!}{(k+1)! \, (k+1)!} \leq 3 \cdot 2^{2k-1}.
\]
\[
\frac{(2k)! \, (2k+1) \, (2k+2)}{k! \, (k+1) \, k! \, (k+1)} \leq 3 \cdot 2^{2k-1}.
\]
\[
\frac{(2k)!}{k!k!} \cdot \frac{(2k+1)(2k+2)}{(k+1)(k+1)} \leq 3 \cdot 2^{2k-1}
\]

\subsection*{Vale decir que}
Por estos calculos realizados arriba
\[
\binom{2(k+1)}{k+1}
\] = \[
\frac{(2k)!}{k!k!} \cdot \frac{(2k+1)(2k+2)}{(k+1)(k+1)}
\]

Por HI 
\[
\frac{(2k)!}{k!k!} \cdot \frac{(2k+1)(2k+2)}{(k+1)(k+1)} \leq 3 \cdot 2^{2k-3}\cdot \frac{(2k+1)(2k+2)}{(k+1)(k+1)}
\] 

Basta con probar que 
\[
3 \cdot 2^{2k-3}\cdot \frac{(2k+1)(2k+2)}{(k+1)(k+1)} \leq 3 \cdot 2^{2k-1}
\] 
\[
2^{2k-3}\cdot \frac{(2k+1)(2k+2)}{(k+1)(k+1)} \leq 2^{2k-1}
\]
\[
2^3\cdot 2^{2k-3}\cdot \frac{(2k+1)(2k+2)}{(k+1)(k+1)} \leq 2^{2k-1}\cdot2^3
\]
\[
2^{2k}\cdot \frac{(2k+1)(2k+2)}{(k+1)(k+1)} \leq 2^{2k+2}
\]
\[
2^{2k}\cdot \frac{(2k+1)(2k+2)}{(k+1)(k+1)} \leq 2^{2k}\cdot2^{2}
\]
\[
\frac{(2k+1)(2k+2)}{(k+1)(k+1)} \leq 2^{2}
\]
\[
(2k+1)(2k+2) \leq 4(k+1)(k+1)
\]
\[
4k^2 + 4k + 2k +2 \leq 4(k^2 +2k + 1)
\]
\[
4k^2 + 6k  +2 \leq 4k^2 +8k + 4
\]
\[
 6k  +2 \leq 8k + 4
\]
\[
-2 \leq 2k
\]
\[
-1 \leq k
\]
Esto es verdadero para todo \( k \in \mathbb{N}\) :D

\subsection*{Concluimos:}

Así, \( p(2) \) es verdadero, \( p(k) \) es verdadero, \( p(k+1) \) es verdadero, con \( k \in \mathbb{N} \) , \( k \geq 2 \). 
\[
\Rightarrow \, p(n) \text{ es verdadero para todo } n \in \mathbb{N}, \, n \geq 2.
\]

\end{document}
